\documentclass[pdftex,12pt,a4paper]{report}

\usepackage[utf8]{inputenc}
\usepackage{amsmath}

\usepackage[sc]{mathpazo}
\linespread{1.05} % Palatino needs more leading (space between lines)
\usepackage[T1]{fontenc}

\usepackage{geometry}
\geometry{margin=1.2in}

\usepackage{caption}

\usepackage{graphicx}
\usepackage{epstopdf}
\usepackage{eurosym}
\usepackage[usenames,dvipsnames]{xcolor}

\setcounter{secnumdepth}{4}
\setcounter{tocdepth}{4}
\renewcommand{\thesection}{\arabic{section}}

\setlength{\parindent}{0pt}

\title{TROFF, just a tron-like.}

\author{Valentin Iovene (toogy)\\ William Fabre (Tchii)\\ Arthur D'Avray\\
Julien Chenglee Miyamoto \and Nicolas Houyengah\\ Sullivan Drouard\\ Karl Mire}


\begin{document}

\maketitle

\newpage

\begin{center}
\includegraphics[width=0.8\textwidth]{pony.jpg}
\end{center}

\newpage

\tableofcontents

\newpage

\section{Tasks distribution}

Valentin Iovene (toogy) had the responsibility of allocating the tasks
to the most appropriate members and in a fair way.

\subsection{So, here it is}

\begin{itemize}
\itemsep1pt\parskip0pt\parsep0pt
\item
  Valentin Iovene: project supervisor, responsible for networking, works
  on almost every part of the project. Also liable for the stupidity of
  his team members
\item
  William Fabre: graphic designer, responsible for the Game-Over
  acknowledgment system
\item
  Arthur D'Avray: responsible for everything related to the movement of
  the motorcycles
\item
  Julien Chenglee Miyamoto: responsible for the trail the motorcycles
  let behind them
\item
  Nicolas Houyengah: responsible for the map. Also beer, café and pizza
  deliverer
\item
  Sullivan Drouard: responsible for the game menu
\item
  Karl Mire: responsible for the ``gameStates'' structure of the game
\end{itemize}

\section{Graphic design}

\textbf{William Fabre}

Let's start with one of the simplest part of this project: graphic
design.

All the graphics you see in this game were made thanks to Photoshop.
Adobe Photoshop is a graphics editing program developed and published by
Adobe Systems. Upon loading Photoshop, a sidebar with a variety of tools
with multiple image-editing functions appears to the left of the screen.
These tools typically fall under the categories of drawing; painting;
measuring and navigation; selection; typing; and retouching. Some tools
contain a small triangle in the bottom right of the toolbox icon. These
can be expanded to reveal similar tools. While newer versions of
Photoshop are updated to include new tools and features, several
recurring tools that exist in most versions are discussed below.

\section{GameStates}

\textbf{Karl Mire}

Every game starts off in an introduction state, then moves to a menu of
some kind, a finally play begins. When you're finally defeated, the game
moves to a game-over state, usually followed by a return to the menu. In
most games it is possible to be in more than one state at a time. For
example, you can usually bring up the menu during game play.

The traditional way of handling multiple states is with a series of if
statements, switches, and loops. The program begins in the intro state
and loops until a key is pressed. Then the menu is displayed until a
selection is made. Then the game begins, and loops until the game is
over. Everytime through the game loop, the program must check to see if
it should display the menu or simply draw the next frame. Also, the part
of the program that handles events must check to see if your input
should affect the menu or the game. All of this combines to make a main
loop that is hard to follow, and therefore hard to debug and maintain.

\section{The menu}

\textbf{Sullivan Drouard}

This menu is composed of menu buttons. Each menu buttons has a different
delegate function triggered when the user presses enter when this button
is focused.

The user can browse the menu thanks to his keyword. Each button may have
a subMenu or may have not.

\section{The game}

\subsection{The map}

\textbf{Nicolas Houyengah}

The map is a matrix of 200*150 bytes. The bytes take a different value
regarding to the state of the map.

When the map is drawn, if the byte value is 0, it does nothing, if it's
1, then it colors the cell in blue (it's player1's trail), if it's 2, it
colors it in yellow (it's player2's trail). If it's some other
byte\ldots{} well\ldots{} we shall consider that we have a problem
(maybe some member of the team is an asshole and don't know how to
code).

\subsection{Players and their trails}

\textbf{Julien Chenglee Miyamoto}

Each player has a position on the map and a direction (North, East,
South and West). This is the only information we need.

Every 10 frames, the player changes its position. When he does so, we
have to change the byte value of the map matrix to the player number.

\subsection{Game-Over acknowledgment system}

\textbf{William Fabre}

Well, if the player's position is out of the born of the map matrix,
he's done. If his new position (just after he moved) is on a cell of the
map with a value different of 0, he's done too. Simple as that

\section{To conclude}

This project was great. The team was great! It probably couldn't be
better. We learned a lot, this kind of exercises is really enriching,
rewarding shall we say. It was a great idea, thank you.

\textbf{In ACDC we can certainly trust.}

\begin{center}
\includegraphics[width=0.5\textwidth]{bullshit.jpg}
\end{center}

\end{document}